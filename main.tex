\DocumentMetadata{
  lang        = de,
  pdfstandard = ua-2,
  pdfstandard = a-4f, %or a-4
  tagging=on,
  tagging-setup={math/setup=mathml-SE} 
}
\documentclass{ltx-talk}

\title{The ltx-talk class}
\subtitle{A replacement for beamer which allows PDF tagging}

\author{Overleaf Support}
\date{\today}

\begin{document}

%The next statement creates the title page.
\begin{frame}
  \maketitle
\end{frame}


%---------------------------------------------------------
%This block of code is for the table of contents after
%the title page
\begin{frame}
\frametitle{Table of Contents}
\tableofcontents
\end{frame}
%---------------------------------------------------------


\section{What is it?}

%---------------------------------------------------------
%Changing visivility of the text
\begin{frame}
\frametitle{ltx-talk on CTAN}
Please see CTAN (\href{https://ctan.org/}{ctan.org}) for more information about \href{https://ctan.org/pkg/ltx-talk}{ltx-talk}.


As mentioned there,
\begin{itemize}
    \item \texttt{ltex-talk} class is focused on producing (on-screen) presentations, along with support material such as handouts and speaker notes.
    \item \texttt{ltex-talk} class has syntax similar to the popular beamer class, although there are some (deliberate) differences. 
    \item \texttt{ltex-talk} has been implemented to support creation of tagged (accessible) PDF output as a core aim. As such, it is suited to creating output for reuse in other formats, e.g. HTML conversions, without additional steps.
\end{itemize}


\end{frame}

\section{How canI use it?}

\begin{frame}
\frametitle{ltx-talk with Overleaf}
\texttt{ltex-talk} is available on Overleaf, but it requires a few special settings.
\begin{itemize}
    \item The \texttt{ltex-talk}class requires \LaTeX 2025-11-01 or later.
    \item Currently, this version is only available on Overleaf through the \textbf{rolling TeX Live} available through \textbf{Overleaf Labs}.
    \item The PDF tagging support requires the use of the LuaLaTeX compiler.
\end{itemize}
\end{frame}


\begin{frame}
\frametitle{A richer example}
A more involved example using \texttt{ltex-talk} is available for download as a zip file from the class author's blog: \href{https://www.texdev.net/2025/07/12/ltx-talk-a-new-class-for-presentations}{https://www.texdev.net/2025/07/12/ltx-talk-a-new-class-for-presentations}.

The zip file provided there can be uploaded to Overleaf to create a project, but the TeX Live version must be updated to be the "rolling" version and the compiler must be updated to be LuaLaTeX.

\end{frame}


\end{document}
